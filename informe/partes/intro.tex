\section{Introducción} % (fold)
\label{sec:introducci_n}

Este informe busca reunir, explicar y ejemplificar las características principales del lenguaje \textbf{Haskell}, desde lo básico de la sintaxis hasta técnicas avanzadas y aplicaciones.

\subsection{Características del lenguaje} % (fold)
\label{sub:caracter_sticas_del_lenguaje}

Haskell es un lenguaje de programación puramente funcional. Decimos que es funcional porque enfatiza la aplicación de funciones y no maneja datos mutables o de estado. Decimos que es puro porque emplea las funciones puras, es decir, que no tienen efectos secundarios, y para una misma entrada siempre se obtiene el mismo resultado. Vamos a ver más adelante que tiene mecanismos para simular funciones impuras (sin ellas no podría comunicarse con el \"mundo real\").

Es nombrado en homenaje al lógico Haskell Curry.

Otras características destacables:

\begin{itemize}
  \item Tipado fuerte (datos de tipo concreto)
  \item Tipado estático (comprobación de tipos durante la compilación)
  \item Inferencia de tipos
  \item Muy alto nivel
\end{itemize}
% subsection caracter_sticas_del_lenguaje (end)

\subsection{Historia} % (fold)
\label{sub:historia}

Los lenguajes funcionales no son algo nuevo: existian hace mucho tiempo. \textit{Lisp} es considerado el primer lenguaje funcional y fue creado a fines de los 50. Pero el interés hacia ellos creció en los 80 con el lanzamiento de \textit{Miranda}, que fue uno de los primeros lenguajes funcionales que apuntaban al uso comercial en vez del uso académico. El problema es que \textit{Miranda} era propietario y pago. En una conferencia de Lenguajes Funcionales en el 87 la conferencia llegó a un consenso para definir un estandar abierto para estos lenguajes.

La primera versión de Haskell se definió en 1990. Siguieron varias definiciones de espicificaciones, que se fueron enumerando (1.0, 1.1, etc.) y a fines del 1997 culminaron en \textbf{Haskell 98}. A partir de ese momento continuó evolucionando. La útlima especificación usada hasta hoy en día es la del \textbf{Haskell 2010}.

% subsection historia (end)

\subsection{Compiladores e interpretes} % (fold)
\label{sub:compiladores_e_interpretes}

El compilador de código Haskell más utilizado es GHC (Glasgow Haskell Compiler). Se utilizó para las pruebas del código para este informe. Algunas de sus características:

\begin{itemize}
  \item Es escrito en Haskell, C y C++.
  \item Es multiplataforma (Se puede utilizar en Windows, Mac OS y la mayoría de sistemas de Unix).
  \item Funciona con la mayoria de arquitecturas de procesador.
  \item Compila a código nativo.
  \item Posee un interprete (GHCi).
  \item Soporta concurrencia y paralelismo.
  \item Posee una gran cantidad de librerías, aunque algunas sólo funcionan bajo GHC.
\end{itemize}

Entre otras implementaciones se pueden destacar los compiladores \textbf{nhc98} y \textbf{HBC}, y el intérprete \textbf{Hugs}.

% subsection compiladores_e_interpretes (end)

% section introducci_n (end)
