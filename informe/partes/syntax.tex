\section{Sintaxis básica} % (fold)
\label{sec:sintaxis_de_funciones}

Las funciones se definen de una manera similar a como son llamadas. El nombre de la función es seguido por los parámetros. Una definición se escribe agregando la definición de lo que hace la función con un \textbf{=} adelante.

\begin{lstlisting}
duplicar x = x + x
\end{lstlisting}

Podemos utilizar la función que acabamos de definir utilizando el interprete de ghci:

\begin{lstlisting}
ghci> duplicar 10
20
ghci> duplicar 3.3
6.6
\end{lstlisting}

Haskell posee de las sentencias \textit{if .. then .. else}. La diferencia de los lenguajes imperativos es que la parte \textit{else} es obligatoria:

\begin{lstlisting}
duplicarNumeroChico x = if x > 100 then x else x * 2
\end{lstlisting}

\subsection{Listas} % (fold)
\label{sub:listas}

Las listas son una estructura de datos muy usada. En Haskell las listas son homogeneas: almacena elementos del mismo tipo.

\begin{lstlisting}
ghci> let numbers = [4,8,15,16,23,42]
ghci> numbers
[4,8,15,16,23,42]
\end{lstlisting}

Las cadenas de texto tambíen son listas: \"hola\" es solo un azucar sintáctico de \lstinline$['H', 'o', 'l', 'a']$. Para concatenar listas en Haskell hacemos:

\begin{lstlisting}
ghci> [1,2,3,4] ++ [9,10,11,12]
[1,2,3,4,9,10,11,12]
ghci> "Hola" ++ " " ++ "mundo"
"Hola mundo"
\end{lstlisting}

Hay que tener cuidado con el operador \textbf{++} ya que para concatenar tiene que recorrer el argumento izquierdo entero. Para concatenar algo al principio de la lista se usa el operador \textbf{:}, es instantáneo:

\begin{lstlisting}
ghci> 'H':"ola mundo"
"Hola mundo"
\end{lstlisting}

Algunas funciones básicas para operar con los elmentos de las listas:

\begin{lstlisting}
ghci> head [1,2,3]
1
ghci> tail [1,2,3]
[2,3]
ghci> last [1,2,3]
3
ghci> reverse [1,2,3]
[3,2,1]
ghci> length [1,2,3]
3
ghci> take 2 [1,2,3]
[1,2]
ghci> drop 2 [1,2,3]
[3]
\end{lstlisting}

\subsubsection{Rangos} % (fold)
\label{ssub:rangos}

¿Qué pasa si queremos todos los números entre 1 y 15? Podríamos escribirlos a mano, pero por suerte Haskell nos simplifica el trabajo con \textit{rangos}. Se utilizan de la siguiente manera:

\begin{lstlisting}
ghci> [1..15]
[1,2,3,4,5,6,7,8,9,10,11,12,13,14,15]
ghci> ['a'..'n']
"abcdefghijklmn"
ghci> [2,4..20]
[2,4,6,8,10,12,14,16,18,20]
\end{lstlisting}

Como se puede ver, funciona con cualquier elemento que tenga alguna orden. En el último ejemplo le estamos especificando al rango un paso.

Los rangos tambien se pueden utilizar para crear listas infinitas. Como ya mencionamos, Haskell es un lenguaje perezoso y solo va a evaluar los elementos de la lista infinita que sean necesarios.

\begin{lstlisting}
ghci> take 5 [1,2..]
[1,2,3,4,5]
\end{lstlisting}

Algunas funciones útiles para la generación de listas infinitas:

\begin{lstlisting}
ghci> take 10 (cycle [1,2,3])
[1,2,3,1,2,3,1,2,3,1]
ghci> take 4 (repeat 'a')
"aaaa"
ghci> replicate "a" 4
"aaaa"
\end{lstlisting}

\subsubsection{Listas por comprensión} % (fold)
\label{ssub:listas_por_comprensi_n}

Algo muy común en las matemáticas es definir conjuntos describiendo como son sus elementos, por ejemplo todos los \textit{x}, \textit{tales que}. En matemáticas podemos definir un conjunto de x naturales, pares y menores a 20 de la siguiente manera: \(S=\{x*2 | x \in N, x < 10\}\). En Haskell lo podriamos escribir como \lstinline$take 10 [2,4..]$, pero, ¿y si queremos algo más complejo? Para ello podemos usar las \textit{listas} por comprensión.

\begin{lstlisting}
ghci> [x*2 | x <- [1..10]]
[2,4,6,8,10,12,14,16,18,20]
\end{lstlisting}

Podemos armar listas de comprensión más comlejas también:

\begin{lstlisting}
ghci> [ x | x <- [50..100], x `mod` 7 == 3]
[52,59,66,73,80,87,94]
ghci> let a = ["foo", "bar"]
ghci> let b = ["123", "456"]
ghci> [x ++ " " ++ y | x <- a, y <- b]
["foo 123","foo 456","bar 123","bar 456"]
\end{lstlisting}

% subsubsection listas_por_comprensi_n (end)

% subsubsection rangos (end)

% subsection listas (end)

\subsection{Tuplas} % (fold)
\label{sub:tuplas}

Tuplas son parecidas a las listas, pero con unas diferencias fundamentales. Las tuplas tienen un número de elementos definido, pero no necesariamente tienen que ser del mismo tipo. A continuación se pueden apreciar algunos ejemplos de uso de tuplas, con las funciones \textbf{fst} y \textbf{snd}, que devuelven el primer y el segundo elemento respectivamente.

\begin{lstlisting}
ghci> fst (1,80)
1
ghci> fst ("Hola", [1,2,3],50)
"Hola"
ghci> snd (1,80)
80
ghci> snd ("Hola", [1,2,3],50)
[1,2,3]
\end{lstlisting}
% subsection tuplas (end)

\subsection{Pattern Matching} % (fold)
\label{sub:pattern_matching}

A continuación se especificarán algunas construcciones sintácticas denominadas como ajuste de patrones (pattern matching).
Una de las ventajas del ajuste de patrones es que se puede simplificar el código de manera que quede más elegante, mejorando la legibilidad.

Un ejemplo simple es crear una función que al pasarle un número verifica si es el número que buscamos:

\begin{lstlisting}
lucky :: (Integral a) => a -> String
lucky 7 = "*'¡El siete de la suerte!'*"
lucky x = "*'Lo siento, ¡no es tu día de suerte!'*"
\end{lstlisting}

Los patrones son verificados de arriba a abajo, de manera que lucky primero verifica si se le pasó el 7 como parámetro a la función, de ser así devuelve una cadena, o de lo contrario devuelve otra.

Otro ejemplo:

\begin{lstlisting}
sayMe :: (Integral a) => a -> String
sayMe 1 = "Uno!"
sayMe 2 = "Dos!"
sayMe 3 = "Tres!"
sayMe x = "No entre uno 1 y 3"
\end{lstlisting}

Otra vez se verifican los patrones de arriba a abajo. Esto es útil, por ejemplo, para no tener que anidar if, then, else en el cuerpo de la función

Si no se cubre algún patrón, al intentar matchear un patrón nuevo, Haskell lanza un error:

\begin{lstlisting}
charName :: Char -> String
charName 'a' = "Alejandro"
charName 'b' = "Bernardo"
charName 'c' = "Carlos"
\end{lstlisting}

Utilizando el intérprete veremos:

\begin{lstlisting}
ghci> charName 'a'
"Alejandro"
ghci> charName 'b'
"Bernardo"
ghci> charName 'c'
"Carlos"
ghci> charName 'z'
:(2,1)-(4,23): Non-exhaustive patterns in function charName
\end{lstlisting}

El patrón \_ indica que no es de importancia y no se utilizará el valor:

\begin{lstlisting}
first :: (a, b, c) -> a
first (x, _, _) = x

second :: (a, b, c) -> b
second (_, y, _) = y

third :: (a, b, c) -> c
third (_, _, z) = z
\end{lstlisting}

El patrón x:xs es utilizado para listas. Diferencian la cabeza (x) del resto de la lista (xs). La sintaxis de este patrón es definida por el caracter :, el cual divide a la cabeza del resto de la lista.

Ejemplos:

Para identificar la cabeza de una lista:

\begin{lstlisting}
head' :: [a] -> a
head' [] = error "*'¡Hey, no puedes utilizar head con una lista vacía!'*"
head' (x:_) = x
\end{lstlisting}

(El patrón [] representa una lista vacía)

Para determinar la longitud de una lista:

\begin{lstlisting}
length' :: (Num b) => [a] -> b
length' [] = 0
length' (_:xs) = 1 + length' xs
\end{lstlisting}

% subsection pattern_matching (end)

\subsection{Guardas} % (fold)
\label{sub:guardas}

Son expresiones que consisten básicamente en booleanos. Si se evalúa a True, se utiliza el cuerpo de la función correspondiente. Son otra buena alternativa para reemplazar un árbol de if, then, else.
Se indican con barras verticales, normalmente tienen sangrías y están alineadas.

Ejemplo:

\begin{lstlisting}
max' :: (Ord a) => a -> a -> a
max' a b
  | a > b = a
  | otherwise = b
\end{lstlisting}

También pueden ser escritas en una sola línea:

\begin{lstlisting}
max' :: (Ord a) => a -> a -> a
max' a b | a > b = a | otherwise = b
\end{lstlisting}

% subsection guardas (end)

\subsection{Operadores Infijos} % (fold)
\label{sub:operadores_infijos}

Los operadores infijos también son funciones en Haskell. Por ejemplo la función de concatenación de listas se define de la siguiente forma:

\begin{lstlisting}
(++)         :: [a] -> [a] -> [a]
[]     ++ ys =  ys
(x:xs) ++ ys =  x : (xs++ys)
\end{lstlisting}

Las funciones comunes tambien pueden ser llamadas de forma infija.

\begin{lstlisting}
add x y = x + y
\end{lstlisting}

\begin{lstlisting}
gchi> 10  `add` 5
15
\end{lstlisting}

% subsection operadores_infijos (end)

% section sintaxis_de_funciones (end)
