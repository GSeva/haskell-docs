\section{Tipos} % (fold)
\label{sec:tipos}

Ya hemos mencionado que Haskell tiene tipado estático y fuerte. Esto significa que por ejemplo si queremos dividir un Boolean por un Int no sólo va a dar error sino que además no va a llegar a compilar.
Al contrario que en Java o en C, donde debemos especificar tipos para que se pueda compilar, en Haskell no siempre es necesario, ya que los tipos, si no se especifican, son inferidos por el compilador.
Para empezar, vemos unos ejemplos simples de tipos, preguntándole al intérprete de GHC:

\begin{lstlisting}
ghci> :t 'a'
'a' :: Char
ghci> :t True
True :: Bool
ghci> :t "HOLA!"
"HOLA!" :: [Char]
ghci> :t (True, 'a')
(True, 'a') :: (Bool, Char)
ghci> :t 4 == 5
4 == 5 :: Bool
\end{lstlisting}


De esta manera vemos que cuando se ejecuta el comando :t en una expresión devuelve esa expresión seguida de :: y su tipo.
En Haskell, las funciones también tienen tipos. Para ver esto vamos a implementar la función declarando primero su tipos. Aquí Haskell no infiere el tipo porque ya está explícito:

\begin{lstlisting}
sumarDos :: Int -> Int
sumarDos a = 2 + a
\end{lstlisting}

Otro ejemplo con 3 entradas:

\begin{lstlisting}
addThree :: Int -> Int -> Int -> Int
addThree x y z = x + y + z
\end{lstlisting}

\textbf{Int} representa enteros. Se utiliza para representar número enteros, por lo que 7 puede ser un \textbf{Int} pero 7.2 no puede. \textbf{Int} está acotado, lo que significa que tiene un valor máximo y un valor mínimo. Normalmente en máquinas
de 32bits el valor máximo de \textbf{Int} es 2147483647 y el mínimo -2147483648.
\textbf{Integer} representa enteros también. La diferencia es que no están acotados así que pueden representar
números muy grandes. Sin embargo, \textbf{Int} es más eficiente.

\begin{lstlisting}
factorial :: Integer -> Integer
factorial n = product [1..n]
\end{lstlisting}

\begin{lstlisting}
ghci> factorial 50
30414093201713378043612608166064768844377641568960512000000000000
\end{lstlisting}

\textbf{Float} es un número real en coma flotante de simple precisión.

\begin{lstlisting}
circumference :: Float -> Float
circumference r = 2 * pi * r
\end{lstlisting}

\begin{lstlisting}
ghci> circumference 4.0
25.132742
\end{lstlisting}

\textbf{Double} es un número real en coma flotante de... ¡Doble precisión!.

\begin{lstlisting}
circumference' :: Double -> Double
circumference' r = 2 * pi * r
\end{lstlisting}

\begin{lstlisting}
ghci> circumference' 4.0
25.132741228718345
\end{lstlisting}

\textbf{Bool} es el tipo booleano. Solo puede tener dos valores: True o False.

\textbf{Char} representa un caracter. Se define rodeado por comillas simples. Una lista de caracteres es una cadena.

\subsection{Variables de tipo} % (fold)
\label{sub:variables_de_tipo}

Una variable de tipo equivale a un tipo genérico. Las funciones que contienen variables de tipo son funciones polimórficas. Esto se explica en el siguiente ejemplo, preguntándole el tipo a la función head. Ejemplo 1:

\begin{lstlisting}
ghci> :t head
head :: forall a. [a] -> a
\end{lstlisting}

\textbf{forall} a indica que a es una a variable de tipo, o sea que para cualquier tipo a, la declaración de tipos de la función head va a ser  \lstinline$[a] -> a$ (recibe una lista que contiene elementos de cualquier tipo y devuelve un elemento de tal tipo).

La siguiente declaración de tipos indica que la función \textbf{fst} recibe una tupla que contiene elementos de cualquier tipo (pueden ser iguales o no), y devuelve un elemento del tipo del primer elemento de la tupla.

\begin{lstlisting}
ghci> fst (1, 2)
1
ghci> :t fst
fst :: forall a b. (a, b) -> a
\end{lstlisting}

% subsection variables_de_tipo (end)

\subsection{Clases de tipo} % (fold)
\label{sub:clases_de_tipo}

% subsection clases_de_tipo (end)

Las clases de tipo tienen un coportamiento parecido a lo que son las interfaces en Java, y en general son utilizadas para generar restricciones para los tipos de las funciones que se declaran. Esto se ve de manera más fácil en el siguiente ejemplo:

Preguntamos el tipo a la función de comparación por igualdad:

\begin{lstlisting}
ghci>:t (==)
(==) :: forall a. Eq a => a -> a -> Bool
\end{lstlisting}

\textbf{(Eq a)} es una restricción en la declaración de tipos.
\textbf{Eq} es una clase de tipo que indica que el tipo genérico \textit{a} es comparable por igualdad.

\textbf{Ord}: Para tipos que poseen algún orden.

\begin{lstlisting}
ghci> :t (>)
(>) :: forall a. Ord a => a -> a -> Bool
\end{lstlisting}

\textbf{(>)} recibe dos parámetros y devuelve \textit{True} si el primero es mayor al segundo. De lo contrario, devuelve \textit{False}.

\textbf{Show}: Para tipos que pueden ser representados en forma de texto.

\begin{lstlisting}
ghci> show 3
"3"
ghci> show False
"False"
ghci> :t show
show :: forall a. Show a => a -> String
\end{lstlisting}

\textbf{show} recibe elementos dentro de la clase de tipo \textbf{Show} y devuelve una cadena de caracteres.

% section tipos (end)
