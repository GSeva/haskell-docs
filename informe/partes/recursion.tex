\section{Recursión} % (fold)
\label{sec:recursi_n}

La recursión o recursividad es la forma de especificar una función basandose
en su propia definición. Es una parte muy importante de Haskell. Una función
es recursiva cuando una parte de su definición incluye a la propia función.
Necesita por lo menos un \textit{caso base} que no hace llamado recursivo para
que exista una condición límite.

Este ejemplo muestra una función de factorial recursiva, separando
claramente el caso base.

\begin{lstlisting}
  factorial 0 = 1
  factorial n = n * factorial (n - 1)
\end{lstlisting}

Muchas de las funciones comunes en Haskell se pueden definir de forma
recursiva, por ejemplo el \textit{length}, la función que devuelve el
número de elementos de una lista:

\begin{lstlisting}
  length []     = 0
  length (x:xs) = 1 + length xs
\end{lstlisting}

Recursión es usada para definir casi todas las funciones de manejo de números
y listas, sin embargo en la práctica no es de usarse tan seguido: la
recursividad está abstraida por las funciones de las librerias de Haskell,
permitiendo al programador razonar sus problemas en más alto nivel. Por
ejemplo, la función de factorial de ejemplo escrita anteriormente se puede
definir de la siguiente manera:

\begin{lstlisting}
  factorial n = product [1..n]
\end{lstlisting}

% section recursi_n (end)


\section{Pereza} % (fold)
\label{sec:pereza}

Una característica destacable de Haskell es que es \textit{perezoso} (o
\textit{no estricto}). Esto significa que nada se va a evaluar hasta que sea
directamente necesario - la evaluación queda diferida hasta que el resultado
es requerido por otra computación. El pasaje de parámetros a las funciones se
realiza por necesidad.

El siguiente es un ejemplo de la pereza de Haskell:

\begin{lstlisting}
  let (a, b) = (length [1..5], reverse "hola mundo") in 1 + 1
\end{lstlisting}

Como la expresión después del \textit{in} no utiliza los valores de a y b,
estos quedan sin evaluar, ya que no son necesarios. Tambien pueden no ser
necesarios por completo, o ser \textit{parcialmente} requeridos:

\begin{lstlisting}
  let (a, b) = (length [1..5], reverse "hola mundo")
      'o':ss = b
\end{lstlisting}

Como solo es necesaria la primera letra para concluir el matcheo de
patrones, la evaluación es parcial.

Las funciones en Haskell pueden ser perezosas o estrictas. Podemos analizar
si una función es perezosa o estricta pasandole \lstinline$undefined$ y viendo
si su ejecución falla. Esto se debe a que en Haskell la evaluación forzada del
\lstinline$undefined$ siempre termina en un error.

\begin{lstlisting}
  let (x, y) = undefined in x  -- Error!
  length [undefined, undefined, undefined]
  -- No hay error, length es perezoso
\end{lstlisting}


La evaluación perezosa tiene muchas ventajas, como la reutilización de código,
posibilidad de generar estructuras de datos infinitas y definiciones
circulares, pero su mayor inconveniente es que el uso de memoria se vuelve muy
dificil de predecir, por ejemplo las expresiones \lstinline$3+2 :: Int$ y
\lstinline$5 :: Int$ denotan el mismo valor pero pueden tener diferentes
tamaños en memoria.
% \singlespace

% \singlespace y
% section pereza (end)
