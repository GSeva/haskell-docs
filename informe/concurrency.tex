

\subsection{Concurrencia} % (fold)
\label{sub:concurrencia}


Un programa concurrente necesita realizar varias tareas al mismo tiempo. Estas
tareas no necesariamente tienen que estar relacionadas entre si. El correcto
funcionamiento de un programa concurrente no necesita varios núcleos.

En contraste, un programa paralelo soluciona un solo problema con el mejor
rendimiento posible, empleando para eso más de un núcleo.

\subsubsection{Threads} % (fold)
\label{ssub:threads}

Un hilo es una acción \textit{IO} que se ejecuta independientemente
de los otros hilos. Los hilos en Haskell no son determinísticos.
Para crear un thread, usamos la función \textit{forkIO} del módulo
\textit{Control.Concurrent}.

Un ejemplo de uso podría ser la compresión de un archivo

% subsubsection threads (end)

% subsection concurrencia (end)


\subsection{Paralelismo} % (fold)
\label{sub:paralelismo}



% subsection paralelismo (end)
